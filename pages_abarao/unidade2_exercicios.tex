\documentclass[12pt,a4paper]{article}
\usepackage[utf8]{inputenc}
\usepackage[T1]{fontenc}
\usepackage[brazil]{babel}
\usepackage{amsmath, amssymb, amsthm}
\usepackage{geometry}
\geometry{margin=2.5cm}
\usepackage{hyperref}
\hypersetup{colorlinks=true,linkcolor=blue,urlcolor=blue}
\usepackage{enumerate}

\title{Lista de Exercícios - Unidade 2\\
\large Convergência Estocástica e Resultados Limite\\
\normalsize 5 Questões por Teorema}
\author{Curso de Inferência Estatística}
\date{Outubro 2025}

\begin{document}

\maketitle
\tableofcontents
\newpage

\section{Introdução}

Esta lista contém 30 questões organizadas por teorema, com 5 questões para cada um dos principais resultados da Unidade 2. Cada questão indica explicitamente qual teorema está sendo testado e utiliza diversas distribuições estudadas no curso.

\textbf{Distribuições utilizadas:} Normal, Poisson, Uniforme, Exponencial, Chi-quadrado, Bernoulli, Cauchy, Gamma e Beta.

\section{Lei Fraca dos Grandes Números}

\subsection*{[Questão 1] Lei Fraca dos Grandes Números}

Sejam $X_1, X_2, \ldots, X_n$ v.a.'s i.i.d. com $X_i \sim N(\mu, \sigma^2)$ onde $\mu = 5$ e $\sigma^2 = 4$.

\begin{enumerate}[(a)]
    \item Mostre que $\bar{X}_n \xrightarrow{P} 5$.
    \item Calcule $P(|\bar{X}_n - 5| \geq 0.5)$ usando a desigualdade de Chebyshev para $n = 64$.
    \item Compare o resultado do item (b) com o valor exato obtido usando que $\bar{X}_n \sim N(5, 4/n)$.
\end{enumerate}

\subsection*{[Questão 2] Lei Fraca dos Grandes Números}

Sejam $X_1, X_2, \ldots, X_n$ v.a.'s i.i.d. com $X_i \sim \text{Poisson}(\lambda)$ onde $\lambda = 3$.

\begin{enumerate}[(a)]
    \item Verifique que $E[X_i]$ e $\text{Var}(X_i)$ são finitos.
    \item Mostre que $\bar{X}_n \xrightarrow{P} 3$.
    \item Encontre $n$ tal que $P(|\bar{X}_n - 3| \geq 0.3) \leq 0.05$ usando a desigualdade de Chebyshev.
\end{enumerate}

\subsection*{[Questão 3] Lei Fraca dos Grandes Números}

Sejam $X_1, X_2, \ldots, X_n$ v.a.'s i.i.d. com $X_i \sim U(0, \theta)$ onde $\theta = 10$.

\begin{enumerate}[(a)]
    \item Calcule $E[X_i]$ e $\text{Var}(X_i)$.
    \item Mostre que $\bar{X}_n \xrightarrow{P} 5$.
    \item Use a LFGN para justificar que $\bar{X}_n$ é um estimador consistente para $\theta/2$.
\end{enumerate}

\subsection*{[Questão 4] Lei Fraca dos Grandes Números}

Sejam $X_1, X_2, \ldots, X_n$ v.a.'s i.i.d. com $X_i \sim \text{Exp}(\beta)$ onde $\beta = 2$ (taxa).

\begin{enumerate}[(a)]
    \item Calcule $E[X_i]$ e $\text{Var}(X_i)$.
    \item Mostre que $\bar{X}_n \xrightarrow{P} 1/2$.
    \item Se quisermos estimar $\beta$ usando $T_n = 1/\bar{X}_n$, mostre que $T_n$ é consistente para $\beta$ usando o teorema da função contínua.
\end{enumerate}

\subsection*{[Questão 5] Lei Fraca dos Grandes Números}

Sejam $X_1, X_2, \ldots, X_n$ v.a.'s i.i.d. com $X_i \sim \text{Cauchy}(0, 1)$ (distribuição de Cauchy padrão).

\begin{enumerate}[(a)]
    \item Explique por que a LFGN \textbf{não pode} ser aplicada diretamente neste caso.
    \item Mostre que $E[|X_i|] = \infty$.
    \item Discuta o comportamento de $\bar{X}_n$ neste caso. Ele converge?
\end{enumerate}

\section{Convergência via Momentos (Resultado 2P)}

\subsection*{[Questão 6] Convergência via Momentos}

Sejam $X_1, X_2, \ldots, X_n$ v.a.'s i.i.d. com $X_i \sim \chi^2_k$ (qui-quadrado com $k$ graus de liberdade).

\begin{enumerate}[(a)]
    \item Mostre que $E[X_i] = k$ e $\text{Var}(X_i) = 2k$.
    \item Use o Resultado 2P com $r = 2$ para mostrar que $\bar{X}_n \xrightarrow{P} k$.
    \item Calcule explicitamente $E[(\bar{X}_n - k)^2]$ e mostre que converge para zero.
\end{enumerate}

\subsection*{[Questão 7] Convergência via Momentos}

Sejam $X_1, X_2, \ldots, X_n$ v.a.'s i.i.d. com $X_i \sim \text{Bernoulli}(p)$ onde $p = 0.6$.

\begin{enumerate}[(a)]
    \item Defina $S_n^2 = \frac{1}{n-1}\sum_{i=1}^n (X_i - \bar{X}_n)^2$.
    \item Mostre que $E[S_n^2] = p(1-p) = 0.24$.
    \item Use o Resultado 2P para mostrar que $S_n^2 \xrightarrow{P} 0.24$.
\end{enumerate}

\subsection*{[Questão 8] Convergência via Momentos}

Sejam $X_1, X_2, \ldots, X_n$ v.a.'s i.i.d. com $X_i \sim U(a, b)$.

\begin{enumerate}[(a)]
    \item Seja $T_n = X_{(n)}$ (o máximo da amostra). Calcule $E[T_n]$ e mostre que $E[T_n] \to b$.
    \item Calcule $E[(T_n - b)^2]$ e mostre que converge para zero.
    \item Conclua que $T_n \xrightarrow{P} b$ pelo Resultado 2P.
\end{enumerate}

\subsection*{[Questão 9] Convergência via Momentos}

Sejam $X_1, X_2, \ldots, X_n$ v.a.'s i.i.d. com $X_i \sim N(\mu, \sigma^2)$.

\begin{enumerate}[(a)]
    \item Seja $T_n = \frac{1}{n}\sum_{i=1}^n X_i^2$. Calcule $E[T_n]$.
    \item Mostre que $E[(T_n - (\mu^2 + \sigma^2))^2] \to 0$.
    \item Conclua que $T_n \xrightarrow{P} \mu^2 + \sigma^2$.
\end{enumerate}

\subsection*{[Questão 10] Convergência via Momentos}

Sejam $X_1, X_2, \ldots, X_n$ v.a.'s i.i.d. com $X_i \sim \text{Exp}(\lambda)$ (taxa $\lambda$).

\begin{enumerate}[(a)]
    \item Defina $T_n = \frac{n}{\sum_{i=1}^n X_i}$. Este é o estimador de máxima verossimilhança para $\lambda$.
    \item Mostre que $E[1/T_n] = 1/\lambda$ (dica: $\sum_{i=1}^n X_i \sim \text{Gamma}(n, \lambda)$).
    \item Argumente que $T_n \xrightarrow{P} \lambda$ usando a LFGN e o teorema da função contínua.
\end{enumerate}

\section{Teorema de Slutsky}

\subsection*{[Questão 11] Teorema de Slutsky}

Sejam $X_1, X_2, \ldots, X_n$ v.a.'s i.i.d. com $X_i \sim N(\mu, \sigma^2)$.

\begin{enumerate}[(a)]
    \item Mostre que $\frac{\sqrt{n}(\bar{X}_n - \mu)}{\sigma} \xrightarrow{D} N(0,1)$ pelo TCL.
    \item Mostre que $S_n \xrightarrow{P} \sigma$.
    \item Use o Teorema de Slutsky para mostrar que $\frac{\sqrt{n}(\bar{X}_n - \mu)}{S_n} \xrightarrow{D} N(0,1)$.
\end{enumerate}

\subsection*{[Questão 12] Teorema de Slutsky}

Sejam $X_1, X_2, \ldots, X_n$ v.a.'s i.i.d. com $X_i \sim \text{Exp}(\lambda)$ onde $\lambda = 2$.

\begin{enumerate}[(a)]
    \item Pelo TCL, $\sqrt{n}(\bar{X}_n - 1/2) \xrightarrow{D} N(0, 1/4)$.
    \item Defina $U_n = \sqrt{n}(\bar{X}_n - 1/2)$ e $V_n = \bar{X}_n$. Mostre que $U_n \xrightarrow{D} N(0, 1/4)$ e $V_n \xrightarrow{P} 1/2$.
    \item Use Slutsky para encontrar a distribuição limite de $W_n = U_n \cdot V_n = \sqrt{n}\bar{X}_n(\bar{X}_n - 1/2)$.
\end{enumerate}

\subsection*{[Questão 13] Teorema de Slutsky}

Sejam $\{X_n, n \geq 1\}$ v.a.'s i.i.d. com $X_i \sim U(0, \theta)$ onde $\theta > 0$ é desconhecido.

\begin{enumerate}[(a)]
    \item Sabe-se que $U_n = \frac{n}{\theta}(\theta - T_n) \xrightarrow{D} \text{Exp}(1)$ onde $T_n = X_{(n)}$.
    \item Mostre que $T_n \xrightarrow{P} \theta$.
    \item Defina $Q_n = \frac{n(\theta - T_n)}{T_n}$. Use Slutsky para encontrar a distribuição limite de $Q_n$.
\end{enumerate}

\subsection*{[Questão 14] Teorema de Slutsky}

Sejam $X_1, X_2, \ldots, X_n$ v.a.'s i.i.d. com $X_i \sim \text{Poisson}(\lambda)$.

\begin{enumerate}[(a)]
    \item Pelo TCL, $\frac{\sqrt{n}(\bar{X}_n - \lambda)}{\sqrt{\lambda}} \xrightarrow{D} N(0,1)$.
    \item Mostre que $\sqrt{\bar{X}_n} \xrightarrow{P} \sqrt{\lambda}$.
    \item Use Slutsky para mostrar que $\frac{\sqrt{n}(\bar{X}_n - \lambda)}{\sqrt{\bar{X}_n}} \xrightarrow{D} N(0,1)$.
\end{enumerate}

\subsection*{[Questão 15] Teorema de Slutsky}

Sejam $X_1, X_2, \ldots, X_n$ v.a.'s i.i.d. com $X_i \sim \text{Cauchy}(\theta, 1)$ (localização $\theta$, escala 1).

\begin{enumerate}[(a)]
    \item Explique por que o TCL não pode ser aplicado diretamente para $\bar{X}_n$.
    \item Suponha que, por outro método, sabemos que $a_n(M_n - \theta) \xrightarrow{D} \text{Cauchy}(0,1)$ onde $M_n$ é a mediana amostral e $a_n$ é alguma constante.
    \item Discuta se seria possível usar Slutsky neste contexto se tivéssemos $V_n \xrightarrow{P} c$.
\end{enumerate}

\section{Teorema Central do Limite}

\subsection*{[Questão 16] Teorema Central do Limite}

Sejam $X_1, X_2, \ldots, X_n$ v.a.'s i.i.d. com $X_i \sim \text{Bernoulli}(p)$ onde $p = 0.3$.

\begin{enumerate}[(a)]
    \item Calcule $E[X_i]$ e $\text{Var}(X_i)$.
    \item Use o TCL para aproximar $P(\bar{X}_n \leq 0.35)$ para $n = 100$.
    \item Compare com a aproximação normal para a binomial: $S_n = \sum_{i=1}^n X_i \sim \text{Binomial}(n, p)$.
\end{enumerate}

\subsection*{[Questão 17] Teorema Central do Limite}

Sejam $X_1, X_2, \ldots, X_n$ v.a.'s i.i.d. com $X_i \sim \text{Exp}(\lambda)$ onde $\lambda = 1$.

\begin{enumerate}[(a)]
    \item Verifique que $E[X_i] = 1$ e $\text{Var}(X_i) = 1$.
    \item Use o TCL para aproximar $P(0.9 \leq \bar{X}_n \leq 1.1)$ para $n = 50$.
    \item Calcule a distribuição exata de $S_n = \sum_{i=1}^n X_i$ e compare.
\end{enumerate}

\subsection*{[Questão 18] Teorema Central do Limite}

Sejam $X_1, X_2, \ldots, X_n$ v.a.'s i.i.d. com $X_i \sim U(0, 1)$.

\begin{enumerate}[(a)]
    \item Calcule $E[X_i] = 1/2$ e $\text{Var}(X_i) = 1/12$.
    \item Use o TCL para encontrar $P\left(\left|\bar{X}_n - \frac{1}{2}\right| \leq 0.05\right)$ para $n = 100$.
    \item Encontre $n$ tal que $P\left(\left|\bar{X}_n - \frac{1}{2}\right| \leq 0.01\right) \geq 0.95$.
\end{enumerate}

\subsection*{[Questão 19] Teorema Central do Limite}

Sejam $X_1, X_2, \ldots, X_n$ v.a.'s i.i.d. com $X_i \sim \text{Poisson}(\lambda)$ onde $\lambda = 5$.

\begin{enumerate}[(a)]
    \item Lembre que para Poisson, $E[X_i] = \text{Var}(X_i) = \lambda = 5$.
    \item Use o TCL para aproximar $P(\bar{X}_n \geq 5.5)$ para $n = 100$.
    \item Use o TCL para aproximar $P(S_n \geq 550)$ onde $S_n = \sum_{i=1}^n X_i$ e compare com o item anterior.
\end{enumerate}

\subsection*{[Questão 20] Teorema Central do Limite}

Sejam $X_1, X_2, \ldots, X_n$ v.a.'s i.i.d. com $X_i \sim \text{Cauchy}(0, 1)$.

\begin{enumerate}[(a)]
    \item Mostre que $E[X_i]$ não existe (integral diverge).
    \item Explique por que o TCL não se aplica.
    \item Pesquise: qual é a distribuição de $\bar{X}_n$ neste caso? (Dica: A soma de Cauchys independentes é Cauchy)
\end{enumerate}

\section{Método Delta / Teorema de Mann-Wald}

\subsection*{[Questão 21] Método Delta}

Sejam $X_1, X_2, \ldots, X_n$ v.a.'s i.i.d. com $X_i \sim N(\mu, \sigma^2)$ onde $\mu > 0$.

\begin{enumerate}[(a)]
    \item Pelo TCL, $\sqrt{n}(\bar{X}_n - \mu) \xrightarrow{D} N(0, \sigma^2)$.
    \item Use o Método Delta com $g(x) = \sqrt{x}$ para encontrar a distribuição assintótica de $\sqrt{n}(\sqrt{\bar{X}_n} - \sqrt{\mu})$.
    \item Calcule $g'(\mu)$ e escreva explicitamente a variância assintótica.
\end{enumerate}

\subsection*{[Questão 22] Método Delta}

Sejam $X_1, X_2, \ldots, X_n$ v.a.'s i.i.d. com $X_i \sim \text{Poisson}(\lambda)$.

\begin{enumerate}[(a)]
    \item Sabe-se que $\sqrt{n}(\bar{X}_n - \lambda) \xrightarrow{D} N(0, \lambda)$.
    \item Use o Método Delta com $g(x) = x^3$ para encontrar a distribuição assintótica de $\sqrt{n}(\bar{X}_n^3 - \lambda^3)$.
    \item Verifique que a variância assintótica é $9\lambda^5$.
\end{enumerate}

\subsection*{[Questão 23] Método Delta}

Sejam $X_1, X_2, \ldots, X_n$ v.a.'s i.i.d. com $X_i \sim \text{Exp}(\lambda)$ (taxa $\lambda$).

\begin{enumerate}[(a)]
    \item Pelo TCL, $\sqrt{n}(\bar{X}_n - 1/\lambda) \xrightarrow{D} N(0, 1/\lambda^2)$.
    \item Use o Método Delta com $g(x) = \log(x)$ para encontrar a distribuição assintótica de $\sqrt{n}(\log(\bar{X}_n) - \log(1/\lambda))$.
    \item Simplifique a variância assintótica.
\end{enumerate}

\subsection*{[Questão 24] Método Delta}

Sejam $X_1, X_2, \ldots, X_n$ v.a.'s i.i.d. com $X_i \sim U(0, 1)$.

\begin{enumerate}[(a)]
    \item Sabemos que $\sqrt{n}(\bar{X}_n - 1/2) \xrightarrow{D} N(0, 1/12)$.
    \item Use o Método Delta com $g(x) = \frac{1}{x}$ para encontrar a distribuição assintótica de $\sqrt{n}\left(\frac{1}{\bar{X}_n} - 2\right)$.
    \item Calcule explicitamente $g'(1/2)$ e a variância assintótica.
\end{enumerate}

\subsection*{[Questão 25] Método Delta}

Sejam $X_1, X_2, \ldots, X_n$ v.a.'s i.i.d. com $X_i \sim \text{Bernoulli}(p)$ onde $0 < p < 1$.

\begin{enumerate}[(a)]
    \item Pelo TCL, $\sqrt{n}(\bar{X}_n - p) \xrightarrow{D} N(0, p(1-p))$.
    \item Use o Método Delta com $g(x) = \log\left(\frac{x}{1-x}\right)$ (transformação logit) para encontrar a distribuição assintótica de $\sqrt{n}\left[\log\left(\frac{\bar{X}_n}{1-\bar{X}_n}\right) - \log\left(\frac{p}{1-p}\right)\right]$.
    \item Calcule $g'(p)$ e verifique que a variância assintótica é $\frac{1}{p(1-p)}$.
\end{enumerate}

\section{Convergência em Distribuição}

\subsection*{[Questão 26] Convergência em Distribuição}

Sejam $X_1, X_2, \ldots, X_n$ v.a.'s i.i.d. com $X_i \sim U(0, \theta)$.

\begin{enumerate}[(a)]
    \item Seja $T_n = X_{(n)}$ o máximo amostral. Encontre a f.d.a. de $T_n$.
    \item Defina $U_n = \frac{n}{\theta}(\theta - T_n)$. Encontre a f.d.a. de $U_n$.
    \item Mostre que $U_n \xrightarrow{D} \text{Exp}(1)$ quando $n \to \infty$.
\end{enumerate}

\subsection*{[Questão 27] Convergência em Distribuição}

Sejam $X_1, X_2, \ldots, X_n$ v.a.'s i.i.d. com $X_i \sim \text{Exp}(1)$.

\begin{enumerate}[(a)]
    \item Seja $Y_n = \min\{X_1, \ldots, X_n\}$. Encontre a distribuição de $Y_n$.
    \item Considere $Z_n = n \cdot Y_n$. Encontre a distribuição de $Z_n$.
    \item O que acontece com a distribuição de $Z_n$ quando $n \to \infty$?
\end{enumerate}

\subsection*{[Questão 28] Convergência em Distribuição}

Sejam $X_1, X_2, \ldots, X_n$ v.a.'s i.i.d. com $X_i \sim N(0, 1)$.

\begin{enumerate}[(a)]
    \item Defina $T_n = \frac{1}{n}\sum_{i=1}^n X_i^2$. Qual é a distribuição de $n \cdot T_n$?
    \item Mostre que $T_n \xrightarrow{P} 1$.
    \item Use o TCL para encontrar a distribuição assintótica de $\sqrt{n}(T_n - 1)$.
\end{enumerate}

\subsection*{[Questão 29] Convergência em Distribuição}

Sejam $Y_n \sim \text{Gamma}(n, n)$ (forma $\alpha = n$, taxa $\beta = n$).

\begin{enumerate}[(a)]
    \item Calcule $E[Y_n]$ e $\text{Var}(Y_n)$.
    \item Mostre que $Y_n \xrightarrow{P} 1$.
    \item Use o TCL para a distribuição Gamma para mostrar que $\sqrt{n}(Y_n - 1) \xrightarrow{D} N(0, 1)$.
\end{enumerate}

\subsection*{[Questão 30] Convergência em Distribuição}

Sejam $X_n \sim \text{Beta}(n, 1)$ para $n \geq 1$.

\begin{enumerate}[(a)]
    \item Encontre a f.d.p. de $X_n$ e mostre que $E[X_n] = \frac{n}{n+1}$.
    \item Mostre que $X_n \xrightarrow{P} 1$.
    \item Defina $Y_n = n(1 - X_n)$. Encontre a distribuição limite de $Y_n$ quando $n \to \infty$.
\end{enumerate}

\section{Gabarito e Dicas}

\subsection{Dicas Gerais de Resolução}

\begin{enumerate}
    \item \textbf{Identifique o teorema aplicável:} Leia atentamente qual teorema está sendo testado no cabeçalho da questão.
    
    \item \textbf{Verifique as condições:} Antes de aplicar um teorema, verifique que todas as condições são satisfeitas (i.i.d., momentos finitos, etc.).
    
    \item \textbf{LFGN:} Use quando precisar mostrar $\bar{X}_n \xrightarrow{P} \mu$. Verifique $E[X_i] < \infty$ e (para versão simples) $\text{Var}(X_i) < \infty$.
    
    \item \textbf{TCL:} Use quando precisar da \emph{distribuição} de $\bar{X}_n$ padronizada. Sempre resulta em $N(0,1)$ assintoticamente.
    
    \item \textbf{Slutsky:} Use quando precisar substituir parâmetros desconhecidos ou combinar convergências de tipos diferentes.
    
    \item \textbf{Método Delta:} Use quando tiver uma transformação não-linear $g(\bar{X}_n)$ e quiser sua distribuição assintótica.
    
    \item \textbf{Distribuição Cauchy:} Lembre-se que é o contraexemplo padrão - não tem momentos finitos!
    
    \item \textbf{Cálculos de variância:} Para $\text{Var}(\bar{X}_n) = \sigma^2/n$. Para soma: $\text{Var}(S_n) = n\sigma^2$.
    
    \item \textbf{Padronização:} Sempre padronize corretamente: $(T_n - E[T_n])/\sqrt{\text{Var}(T_n)}$.
    
    \item \textbf{Derivadas no Método Delta:} Não esqueça de calcular $g'(\theta)$ e elevar ao quadrado para a variância.
\end{enumerate}

\subsection{Respostas Selecionadas}

\textbf{Questão 5(c):} $\bar{X}_n$ tem a mesma distribuição que $X_1$ (distribuição Cauchy) para todo $n$. Não há convergência!

\textbf{Questão 11(c):} Este é o resultado fundamental que permite usar estatística $t$ quando $\sigma$ é desconhecido.

\textbf{Questão 16(b):} $P(\bar{X}_n \leq 0.35) \approx P\left(Z \leq \frac{0.35 - 0.3}{\sqrt{0.21/100}}\right) = P(Z \leq 1.09) \approx 0.862$.

\textbf{Questão 21(b):} Variância assintótica: $\sigma^2/(4\mu)$.

\textbf{Questão 26(c):} Use que $\lim_{n \to \infty} \left(1 - \frac{u}{n}\right)^n = e^{-u}$.

\textbf{Questão 30(c):} $Y_n \xrightarrow{D} \text{Exp}(1)$ (use transformação de variáveis).

\end{document}

