\documentclass[12pt,a4paper]{article}
\usepackage[utf8]{inputenc}
\usepackage[T1]{fontenc}
\usepackage[brazil]{babel}
\usepackage{amsmath, amssymb, amsthm}
\usepackage{geometry}
\geometry{margin=2.5cm}
\usepackage{hyperref}
\hypersetup{colorlinks=true,linkcolor=blue,urlcolor=blue}
\usepackage{enumerate}

\title{Lista de Exercícios - Unidade 2\\
\large Convergência Estocástica e Resultados Limite\\
\normalsize 50 Questões Completas}
\author{Curso de Inferência Estatística}
\date{Outubro 2025 - Versão Atualizada}

\begin{document}

\maketitle
\tableofcontents
\newpage

\section{Introdução}

Esta lista contém 50 questões organizadas por teorema, com 5 questões para cada um dos principais resultados da Unidade 2. Cada questão indica explicitamente qual teorema está sendo testado e utiliza diversas distribuições estudadas no curso.

\textbf{Distribuições utilizadas:} Normal, Poisson, Uniforme, Exponencial, Chi-quadrado, Bernoulli, Cauchy, Gamma e Beta.

\textbf{Teoremas cobertos:}
\begin{enumerate}
    \item Lei Fraca dos Grandes Números
    \item Convergência via Momentos
    \item Teorema de Slutsky
    \item Teorema Central do Limite
    \item Método Delta / Teorema de Mann-Wald
    \item Convergência em Distribuição
    \item TCL para Variância Amostral
    \item Teorema da Função Contínua (Convergência em Distribuição)
    \item Estimadores Consistentes
    \item Propriedades Assintóticas dos EMVs
\end{enumerate}

\section{Lei Fraca dos Grandes Números}

\subsection*{[Questão 1] Lei Fraca dos Grandes Números}

Sejam $X_1, X_2, \ldots, X_n$ v.a.'s i.i.d. com $X_i \sim N(\mu, \sigma^2)$ onde $\mu = 5$ e $\sigma^2 = 4$.

\begin{enumerate}[(a)]
    \item Mostre que $\bar{X}_n \xrightarrow{P} 5$.
    \item Calcule $P(|\bar{X}_n - 5| \geq 0.5)$ usando a desigualdade de Chebyshev para $n = 64$.
    \item Compare o resultado do item (b) com o valor exato obtido usando que $\bar{X}_n \sim N(5, 4/n)$.
\end{enumerate}

\subsection*{[Questão 2] Lei Fraca dos Grandes Números}

Sejam $X_1, X_2, \ldots, X_n$ v.a.'s i.i.d. com $X_i \sim \text{Poisson}(\lambda)$ onde $\lambda = 3$.

\begin{enumerate}[(a)]
    \item Verifique que $E[X_i]$ e $\text{Var}(X_i)$ são finitos.
    \item Mostre que $\bar{X}_n \xrightarrow{P} 3$.
    \item Encontre $n$ tal que $P(|\bar{X}_n - 3| \geq 0.3) \leq 0.05$ usando a desigualdade de Chebyshev.
\end{enumerate}

\subsection*{[Questão 3] Lei Fraca dos Grandes Números}

Sejam $X_1, X_2, \ldots, X_n$ v.a.'s i.i.d. com $X_i \sim U(0, \theta)$ onde $\theta = 10$.

\begin{enumerate}[(a)]
    \item Calcule $E[X_i]$ e $\text{Var}(X_i)$.
    \item Mostre que $\bar{X}_n \xrightarrow{P} 5$.
    \item Use a LFGN para justificar que $\bar{X}_n$ é um estimador consistente para $\theta/2$.
\end{enumerate}

\subsection*{[Questão 4] Lei Fraca dos Grandes Números}

Sejam $X_1, X_2, \ldots, X_n$ v.a.'s i.i.d. com $X_i \sim \text{Exp}(\beta)$ onde $\beta = 2$ (taxa).

\begin{enumerate}[(a)]
    \item Calcule $E[X_i]$ e $\text{Var}(X_i)$.
    \item Mostre que $\bar{X}_n \xrightarrow{P} 1/2$.
    \item Se quisermos estimar $\beta$ usando $T_n = 1/\bar{X}_n$, mostre que $T_n$ é consistente para $\beta$ usando o teorema da função contínua.
\end{enumerate}

\subsection*{[Questão 5] Lei Fraca dos Grandes Números}

Sejam $X_1, X_2, \ldots, X_n$ v.a.'s i.i.d. com $X_i \sim \text{Cauchy}(0, 1)$ (distribuição de Cauchy padrão).

\begin{enumerate}[(a)]
    \item Explique por que a LFGN \textbf{não pode} ser aplicada diretamente neste caso.
    \item Mostre que $E[|X_i|] = \infty$.
    \item Discuta o comportamento de $\bar{X}_n$ neste caso. Ele converge?
\end{enumerate}

\section{Convergência via Momentos (Resultado 2P)}

\subsection*{[Questão 6] Convergência via Momentos}

Sejam $X_1, X_2, \ldots, X_n$ v.a.'s i.i.d. com $X_i \sim \chi^2_k$ (qui-quadrado com $k$ graus de liberdade).

\begin{enumerate}[(a)]
    \item Mostre que $E[X_i] = k$ e $\text{Var}(X_i) = 2k$.
    \item Use o Resultado 2P com $r = 2$ para mostrar que $\bar{X}_n \xrightarrow{P} k$.
    \item Calcule explicitamente $E[(\bar{X}_n - k)^2]$ e mostre que converge para zero.
\end{enumerate}

\subsection*{[Questão 7] Convergência via Momentos}

Sejam $X_1, X_2, \ldots, X_n$ v.a.'s i.i.d. com $X_i \sim \text{Bernoulli}(p)$ onde $p = 0.6$.

\begin{enumerate}[(a)]
    \item Defina $S_n^2 = \frac{1}{n-1}\sum_{i=1}^n (X_i - \bar{X}_n)^2$.
    \item Mostre que $E[S_n^2] = p(1-p) = 0.24$.
    \item Use o Resultado 2P para mostrar que $S_n^2 \xrightarrow{P} 0.24$.
\end{enumerate}

\subsection*{[Questão 8] Convergência via Momentos}

Sejam $X_1, X_2, \ldots, X_n$ v.a.'s i.i.d. com $X_i \sim U(a, b)$.

\begin{enumerate}[(a)]
    \item Seja $T_n = X_{(n)}$ (o máximo da amostra). Calcule $E[T_n]$ e mostre que $E[T_n] \to b$.
    \item Calcule $E[(T_n - b)^2]$ e mostre que converge para zero.
    \item Conclua que $T_n \xrightarrow{P} b$ pelo Resultado 2P.
\end{enumerate}

\subsection*{[Questão 9] Convergência via Momentos}

Sejam $X_1, X_2, \ldots, X_n$ v.a.'s i.i.d. com $X_i \sim N(\mu, \sigma^2)$.

\begin{enumerate}[(a)]
    \item Seja $T_n = \frac{1}{n}\sum_{i=1}^n X_i^2$. Calcule $E[T_n]$.
    \item Mostre que $E[(T_n - (\mu^2 + \sigma^2))^2] \to 0$.
    \item Conclua que $T_n \xrightarrow{P} \mu^2 + \sigma^2$.
\end{enumerate}

\subsection*{[Questão 10] Convergência via Momentos}

Sejam $X_1, X_2, \ldots, X_n$ v.a.'s i.i.d. com $X_i \sim \text{Exp}(\lambda)$ (taxa $\lambda$).

\begin{enumerate}[(a)]
    \item Defina $T_n = \frac{n}{\sum_{i=1}^n X_i}$. Este é o estimador de máxima verossimilhança para $\lambda$.
    \item Mostre que $E[1/T_n] = 1/\lambda$ (dica: $\sum_{i=1}^n X_i \sim \text{Gamma}(n, \lambda)$).
    \item Argumente que $T_n \xrightarrow{P} \lambda$ usando a LFGN e o teorema da função contínua.
\end{enumerate}

\section{Teorema de Slutsky}

\subsection*{[Questão 11] Teorema de Slutsky}

Sejam $X_1, X_2, \ldots, X_n$ v.a.'s i.i.d. com $X_i \sim N(\mu, \sigma^2)$.

\begin{enumerate}[(a)]
    \item Mostre que $\frac{\sqrt{n}(\bar{X}_n - \mu)}{\sigma} \xrightarrow{D} N(0,1)$ pelo TCL.
    \item Mostre que $S_n \xrightarrow{P} \sigma$.
    \item Use o Teorema de Slutsky para mostrar que $\frac{\sqrt{n}(\bar{X}_n - \mu)}{S_n} \xrightarrow{D} N(0,1)$.
\end{enumerate}

\subsection*{[Questão 12] Teorema de Slutsky}

Sejam $X_1, X_2, \ldots, X_n$ v.a.'s i.i.d. com $X_i \sim \text{Exp}(\lambda)$ onde $\lambda = 2$.

\begin{enumerate}[(a)]
    \item Pelo TCL, $\sqrt{n}(\bar{X}_n - 1/2) \xrightarrow{D} N(0, 1/4)$.
    \item Defina $U_n = \sqrt{n}(\bar{X}_n - 1/2)$ e $V_n = \bar{X}_n$. Mostre que $U_n \xrightarrow{D} N(0, 1/4)$ e $V_n \xrightarrow{P} 1/2$.
    \item Use Slutsky para encontrar a distribuição limite de $W_n = U_n \cdot V_n = \sqrt{n}\bar{X}_n(\bar{X}_n - 1/2)$.
\end{enumerate}

\subsection*{[Questão 13] Teorema de Slutsky}

Sejam $\{X_n, n \geq 1\}$ v.a.'s i.i.d. com $X_i \sim U(0, \theta)$ onde $\theta > 0$ é desconhecido.

\begin{enumerate}[(a)]
    \item Sabe-se que $U_n = \frac{n}{\theta}(\theta - T_n) \xrightarrow{D} \text{Exp}(1)$ onde $T_n = X_{(n)}$.
    \item Mostre que $T_n \xrightarrow{P} \theta$.
    \item Defina $Q_n = \frac{n(\theta - T_n)}{T_n}$. Use Slutsky para encontrar a distribuição limite de $Q_n$.
\end{enumerate}

\subsection*{[Questão 14] Teorema de Slutsky}

Sejam $X_1, X_2, \ldots, X_n$ v.a.'s i.i.d. com $X_i \sim \text{Poisson}(\lambda)$.

\begin{enumerate}[(a)]
    \item Pelo TCL, $\frac{\sqrt{n}(\bar{X}_n - \lambda)}{\sqrt{\lambda}} \xrightarrow{D} N(0,1)$.
    \item Mostre que $\sqrt{\bar{X}_n} \xrightarrow{P} \sqrt{\lambda}$.
    \item Use Slutsky para mostrar que $\frac{\sqrt{n}(\bar{X}_n - \lambda)}{\sqrt{\bar{X}_n}} \xrightarrow{D} N(0,1)$.
\end{enumerate}

\subsection*{[Questão 15] Teorema de Slutsky}

Sejam $X_1, X_2, \ldots, X_n$ v.a.'s i.i.d. com $X_i \sim \text{Cauchy}(\theta, 1)$ (localização $\theta$, escala 1).

\begin{enumerate}[(a)]
    \item Explique por que o TCL não pode ser aplicado diretamente para $\bar{X}_n$.
    \item Suponha que, por outro método, sabemos que $a_n(M_n - \theta) \xrightarrow{D} \text{Cauchy}(0,1)$ onde $M_n$ é a mediana amostral e $a_n$ é alguma constante.
    \item Discuta se seria possível usar Slutsky neste contexto se tivéssemos $V_n \xrightarrow{P} c$.
\end{enumerate}

\section{Teorema Central do Limite}

\subsection*{[Questão 16] Teorema Central do Limite}

Sejam $X_1, X_2, \ldots, X_n$ v.a.'s i.i.d. com $X_i \sim \text{Bernoulli}(p)$ onde $p = 0.3$.

\begin{enumerate}[(a)]
    \item Calcule $E[X_i]$ e $\text{Var}(X_i)$.
    \item Use o TCL para aproximar $P(\bar{X}_n \leq 0.35)$ para $n = 100$.
    \item Compare com a aproximação normal para a binomial: $S_n = \sum_{i=1}^n X_i \sim \text{Binomial}(n, p)$.
\end{enumerate}

\subsection*{[Questão 17] Teorema Central do Limite}

Sejam $X_1, X_2, \ldots, X_n$ v.a.'s i.i.d. com $X_i \sim \text{Exp}(\lambda)$ onde $\lambda = 1$.

\begin{enumerate}[(a)]
    \item Verifique que $E[X_i] = 1$ e $\text{Var}(X_i) = 1$.
    \item Use o TCL para aproximar $P(0.9 \leq \bar{X}_n \leq 1.1)$ para $n = 50$.
    \item Calcule a distribuição exata de $S_n = \sum_{i=1}^n X_i$ e compare.
\end{enumerate}

\subsection*{[Questão 18] Teorema Central do Limite}

Sejam $X_1, X_2, \ldots, X_n$ v.a.'s i.i.d. com $X_i \sim U(0, 1)$.

\begin{enumerate}[(a)]
    \item Calcule $E[X_i] = 1/2$ e $\text{Var}(X_i) = 1/12$.
    \item Use o TCL para encontrar $P\left(\left|\bar{X}_n - \frac{1}{2}\right| \leq 0.05\right)$ para $n = 100$.
    \item Encontre $n$ tal que $P\left(\left|\bar{X}_n - \frac{1}{2}\right| \leq 0.01\right) \geq 0.95$.
\end{enumerate}

\subsection*{[Questão 19] Teorema Central do Limite}

Sejam $X_1, X_2, \ldots, X_n$ v.a.'s i.i.d. com $X_i \sim \text{Poisson}(\lambda)$ onde $\lambda = 5$.

\begin{enumerate}[(a)]
    \item Lembre que para Poisson, $E[X_i] = \text{Var}(X_i) = \lambda = 5$.
    \item Use o TCL para aproximar $P(\bar{X}_n \geq 5.5)$ para $n = 100$.
    \item Use o TCL para aproximar $P(S_n \geq 550)$ onde $S_n = \sum_{i=1}^n X_i$ e compare com o item anterior.
\end{enumerate}

\subsection*{[Questão 20] Teorema Central do Limite}

Sejam $X_1, X_2, \ldots, X_n$ v.a.'s i.i.d. com $X_i \sim \text{Cauchy}(0, 1)$.

\begin{enumerate}[(a)]
    \item Mostre que $E[X_i]$ não existe (integral diverge).
    \item Explique por que o TCL não se aplica.
    \item Pesquise: qual é a distribuição de $\bar{X}_n$ neste caso? (Dica: A soma de Cauchys independentes é Cauchy)
\end{enumerate}

\section{Método Delta / Teorema de Mann-Wald}

\subsection*{[Questão 21] Método Delta}

Sejam $X_1, X_2, \ldots, X_n$ v.a.'s i.i.d. com $X_i \sim N(\mu, \sigma^2)$ onde $\mu > 0$.

\begin{enumerate}[(a)]
    \item Pelo TCL, $\sqrt{n}(\bar{X}_n - \mu) \xrightarrow{D} N(0, \sigma^2)$.
    \item Use o Método Delta com $g(x) = \sqrt{x}$ para encontrar a distribuição assintótica de $\sqrt{n}(\sqrt{\bar{X}_n} - \sqrt{\mu})$.
    \item Calcule $g'(\mu)$ e escreva explicitamente a variância assintótica.
\end{enumerate}

\subsection*{[Questão 22] Método Delta}

Sejam $X_1, X_2, \ldots, X_n$ v.a.'s i.i.d. com $X_i \sim \text{Poisson}(\lambda)$.

\begin{enumerate}[(a)]
    \item Sabe-se que $\sqrt{n}(\bar{X}_n - \lambda) \xrightarrow{D} N(0, \lambda)$.
    \item Use o Método Delta com $g(x) = x^3$ para encontrar a distribuição assintótica de $\sqrt{n}(\bar{X}_n^3 - \lambda^3)$.
    \item Verifique que a variância assintótica é $9\lambda^5$.
\end{enumerate}

\subsection*{[Questão 23] Método Delta}

Sejam $X_1, X_2, \ldots, X_n$ v.a.'s i.i.d. com $X_i \sim \text{Exp}(\lambda)$ (taxa $\lambda$).

\begin{enumerate}[(a)]
    \item Pelo TCL, $\sqrt{n}(\bar{X}_n - 1/\lambda) \xrightarrow{D} N(0, 1/\lambda^2)$.
    \item Use o Método Delta com $g(x) = \log(x)$ para encontrar a distribuição assintótica de $\sqrt{n}(\log(\bar{X}_n) - \log(1/\lambda))$.
    \item Simplifique a variância assintótica.
\end{enumerate}

\subsection*{[Questão 24] Método Delta}

Sejam $X_1, X_2, \ldots, X_n$ v.a.'s i.i.d. com $X_i \sim U(0, 1)$.

\begin{enumerate}[(a)]
    \item Sabemos que $\sqrt{n}(\bar{X}_n - 1/2) \xrightarrow{D} N(0, 1/12)$.
    \item Use o Método Delta com $g(x) = \frac{1}{x}$ para encontrar a distribuição assintótica de $\sqrt{n}\left(\frac{1}{\bar{X}_n} - 2\right)$.
    \item Calcule explicitamente $g'(1/2)$ e a variância assintótica.
\end{enumerate}

\subsection*{[Questão 25] Método Delta}

Sejam $X_1, X_2, \ldots, X_n$ v.a.'s i.i.d. com $X_i \sim \text{Bernoulli}(p)$ onde $0 < p < 1$.

\begin{enumerate}[(a)]
    \item Pelo TCL, $\sqrt{n}(\bar{X}_n - p) \xrightarrow{D} N(0, p(1-p))$.
    \item Use o Método Delta com $g(x) = \log\left(\frac{x}{1-x}\right)$ (transformação logit) para encontrar a distribuição assintótica de $\sqrt{n}\left[\log\left(\frac{\bar{X}_n}{1-\bar{X}_n}\right) - \log\left(\frac{p}{1-p}\right)\right]$.
    \item Calcule $g'(p)$ e verifique que a variância assintótica é $\frac{1}{p(1-p)}$.
\end{enumerate}

\section{Convergência em Distribuição}

\subsection*{[Questão 26] Convergência em Distribuição}

Sejam $X_1, X_2, \ldots, X_n$ v.a.'s i.i.d. com $X_i \sim U(0, \theta)$.

\begin{enumerate}[(a)]
    \item Seja $T_n = X_{(n)}$ o máximo amostral. Encontre a f.d.a. de $T_n$.
    \item Defina $U_n = \frac{n}{\theta}(\theta - T_n)$. Encontre a f.d.a. de $U_n$.
    \item Mostre que $U_n \xrightarrow{D} \text{Exp}(1)$ quando $n \to \infty$.
\end{enumerate}

\subsection*{[Questão 27] Convergência em Distribuição}

Sejam $X_1, X_2, \ldots, X_n$ v.a.'s i.i.d. com $X_i \sim \text{Exp}(1)$.

\begin{enumerate}[(a)]
    \item Seja $Y_n = \min\{X_1, \ldots, X_n\}$. Encontre a distribuição de $Y_n$.
    \item Considere $Z_n = n \cdot Y_n$. Encontre a distribuição de $Z_n$.
    \item O que acontece com a distribuição de $Z_n$ quando $n \to \infty$?
\end{enumerate}

\subsection*{[Questão 28] Convergência em Distribuição}

Sejam $X_1, X_2, \ldots, X_n$ v.a.'s i.i.d. com $X_i \sim N(0, 1)$.

\begin{enumerate}[(a)]
    \item Defina $T_n = \frac{1}{n}\sum_{i=1}^n X_i^2$. Qual é a distribuição de $n \cdot T_n$?
    \item Mostre que $T_n \xrightarrow{P} 1$.
    \item Use o TCL para encontrar a distribuição assintótica de $\sqrt{n}(T_n - 1)$.
\end{enumerate}

\subsection*{[Questão 29] Convergência em Distribuição}

Sejam $Y_n \sim \text{Gamma}(n, n)$ (forma $\alpha = n$, taxa $\beta = n$).

\begin{enumerate}[(a)]
    \item Calcule $E[Y_n]$ e $\text{Var}(Y_n)$.
    \item Mostre que $Y_n \xrightarrow{P} 1$.
    \item Use o TCL para a distribuição Gamma para mostrar que $\sqrt{n}(Y_n - 1) \xrightarrow{D} N(0, 1)$.
\end{enumerate}

\subsection*{[Questão 30] Convergência em Distribuição}

Sejam $X_n \sim \text{Beta}(n, 1)$ para $n \geq 1$.

\begin{enumerate}[(a)]
    \item Encontre a f.d.p. de $X_n$ e mostre que $E[X_n] = \frac{n}{n+1}$.
    \item Mostre que $X_n \xrightarrow{P} 1$.
    \item Defina $Y_n = n(1 - X_n)$. Encontre a distribuição limite de $Y_n$ quando $n \to \infty$.
\end{enumerate}

\section{Gabarito e Dicas}

\subsection{Dicas Gerais de Resolução}

\begin{enumerate}
    \item \textbf{Identifique o teorema aplicável:} Leia atentamente qual teorema está sendo testado no cabeçalho da questão.
    
    \item \textbf{Verifique as condições:} Antes de aplicar um teorema, verifique que todas as condições são satisfeitas (i.i.d., momentos finitos, etc.).
    
    \item \textbf{LFGN:} Use quando precisar mostrar $\bar{X}_n \xrightarrow{P} \mu$. Verifique $E[X_i] < \infty$ e (para versão simples) $\text{Var}(X_i) < \infty$.
    
    \item \textbf{TCL:} Use quando precisar da \emph{distribuição} de $\bar{X}_n$ padronizada. Sempre resulta em $N(0,1)$ assintoticamente.
    
    \item \textbf{Slutsky:} Use quando precisar substituir parâmetros desconhecidos ou combinar convergências de tipos diferentes.
    
    \item \textbf{Método Delta:} Use quando tiver uma transformação não-linear $g(\bar{X}_n)$ e quiser sua distribuição assintótica.
    
    \item \textbf{Distribuição Cauchy:} Lembre-se que é o contraexemplo padrão - não tem momentos finitos!
    
    \item \textbf{Cálculos de variância:} Para $\text{Var}(\bar{X}_n) = \sigma^2/n$. Para soma: $\text{Var}(S_n) = n\sigma^2$.
    
    \item \textbf{Padronização:} Sempre padronize corretamente: $(T_n - E[T_n])/\sqrt{\text{Var}(T_n)}$.
    
    \item \textbf{Derivadas no Método Delta:} Não esqueça de calcular $g'(\theta)$ e elevar ao quadrado para a variância.
\end{enumerate}

\subsection{Respostas Selecionadas}

\textbf{Questão 5(c):} $\bar{X}_n$ tem a mesma distribuição que $X_1$ (distribuição Cauchy) para todo $n$. Não há convergência!

\textbf{Questão 11(c):} Este é o resultado fundamental que permite usar estatística $t$ quando $\sigma$ é desconhecido.

\textbf{Questão 16(b):} $P(\bar{X}_n \leq 0.35) \approx P\left(Z \leq \frac{0.35 - 0.3}{\sqrt{0.21/100}}\right) = P(Z \leq 1.09) \approx 0.862$.

\textbf{Questão 21(b):} Variância assintótica: $\sigma^2/(4\mu)$.

\textbf{Questão 26(c):} Use que $\lim_{n \to \infty} \left(1 - \frac{u}{n}\right)^n = e^{-u}$.

\textbf{Questão 30(c):} $Y_n \xrightarrow{D} \text{Exp}(1)$ (use transformação de variáveis).

\section{TCL para Variância Amostral}

\subsection*{[Questão 31] TCL para Variância Amostral}

Sejam $X_1, X_2, \ldots, X_n$ v.a.'s i.i.d. com $X_i \sim N(\mu, \sigma^2)$ onde $\mu = 0$ e $\sigma^2 = 1$.

\begin{enumerate}[(a)]
    \item Calcule $\mu_4 = E[(X_i - \mu)^4]$ para a distribuição normal padrão.
    \item Use o TCL para variância amostral para encontrar a distribuição assintótica de $\sqrt{n}(S_n^2 - 1)$.
    \item Construa um intervalo de confiança assintótico de 95\% para $\sigma^2$ quando $n = 100$ e $S_n^2 = 1.2$.
\end{enumerate}

\subsection*{[Questão 32] TCL para Variância Amostral}

Sejam $X_1, X_2, \ldots, X_n$ v.a.'s i.i.d. com $X_i \sim \text{Exp}(\lambda)$ onde $\lambda = 2$.

\begin{enumerate}[(a)]
    \item Calcule $E[X_i] = 1/2$, $\text{Var}(X_i) = 1/4$, e $\mu_4 = E[(X_i - 1/2)^4]$.
    \item Dica: Para exponencial, $\mu_4 = 9/\lambda^4$. Verifique este valor.
    \item Use o TCL para $S_n^2$ para encontrar $P(S_n^2 > 0.3)$ aproximadamente quando $n = 50$.
\end{enumerate}

\subsection*{[Questão 33] TCL para Variância Amostral}

Sejam $X_1, X_2, \ldots, X_n$ v.a.'s i.i.d. com $X_i \sim U(0, 1)$.

\begin{enumerate}[(a)]
    \item Calcule $E[X_i] = 1/2$, $\text{Var}(X_i) = 1/12$, e o quarto momento central.
    \item Mostre que $\mu_4 = E[(X_i - 1/2)^4] = 1/80$.
    \item Use o TCL para $S_n^2$ para aproximar $P(|S_n^2 - 1/12| \leq 0.01)$ quando $n = 100$.
\end{enumerate}

\subsection*{[Questão 34] TCL para Variância Amostral}

Sejam $X_1, X_2, \ldots, X_n$ v.a.'s i.i.d. com $X_i \sim \text{Poisson}(\lambda)$ onde $\lambda = 5$.

\begin{enumerate}[(a)]
    \item Para Poisson, mostre que $\mu_4 = \lambda + 3\lambda^2 + \lambda^3 = 5 + 75 + 125 = 205$.
    \item Use o TCL para variância amostral: $\sqrt{n}(S_n^2 - \lambda) \xrightarrow{d} N(0, \mu_4 - \lambda^2)$.
    \item Calcule a variância assintótica e use-a para aproximar $P(S_n^2 > 6)$ quando $n = 80$.
\end{enumerate}

\subsection*{[Questão 35] TCL para Variância Amostral}

Sejam $X_1, X_2, \ldots, X_n$ v.a.'s i.i.d. com $X_i \sim \text{Bernoulli}(p)$ onde $p = 0.4$.

\begin{enumerate}[(a)]
    \item Calcule $\text{Var}(X_i) = p(1-p) = 0.24$ e $\mu_4 = p(1-p)[(1-p)^2 + p^2]$.
    \item Simplifique: $\mu_4 = p(1-p)(1-2p(1-p))$ e calcule para $p = 0.4$.
    \item Use o TCL para $S_n^2$ para testar $H_0: \sigma^2 = 0.24$ vs $H_1: \sigma^2 \neq 0.24$ ao nível 5\% quando $n = 100$ e $S_n^2 = 0.28$.
\end{enumerate}

\section{Teorema da Função Contínua para Convergência em Distribuição}

\subsection*{[Questão 36] Teorema da Função Contínua (Distribuição)}

Sejam $X_1, X_2, \ldots, X_n$ v.a.'s i.i.d. com $X_i \sim N(\mu, \sigma^2)$.

\begin{enumerate}[(a)]
    \item Pelo TCL, $Z_n = \frac{\sqrt{n}(\bar{X}_n - \mu)}{\sigma} \xrightarrow{d} N(0,1)$.
    \item Use o teorema da função contínua com $g(x) = x^2$ para mostrar que $Z_n^2 \xrightarrow{d} \chi^2_1$.
    \item Conclua que $n\left(\frac{\bar{X}_n - \mu}{\sigma}\right)^2 \xrightarrow{d} \chi^2_1$.
\end{enumerate}

\subsection*{[Questão 37] Teorema da Função Contínua (Distribuição)}

Sejam $X_1, X_2, \ldots, X_n$ v.a.'s i.i.d. com $X_i \sim \text{Exp}(\lambda)$.

\begin{enumerate}[(a)]
    \item Pelo TCL, $\frac{\sqrt{n}(\bar{X}_n - 1/\lambda)}{\sqrt{1/\lambda^2}} \xrightarrow{d} N(0,1)$.
    \item Defina $Z_n = \lambda\sqrt{n}(\bar{X}_n - 1/\lambda)$. Use o teorema da função contínua com $g(x) = |x|$ para encontrar a distribuição de $|Z_n|$.
    \item Mostre que $Z_n^2 \xrightarrow{d} \chi^2_1$.
\end{enumerate}

\subsection*{[Questão 38] Teorema da Função Contínua (Distribuição)}

Sejam $X_1, X_2, \ldots, X_n$ v.a.'s i.i.d. com $X_i \sim U(0,1)$.

\begin{enumerate}[(a)]
    \item Sabemos que $\sqrt{12n}(\bar{X}_n - 1/2) \xrightarrow{d} N(0,1)$.
    \item Use o teorema da função contínua com $g(x) = e^x$ para encontrar a distribuição limite de $\exp(\sqrt{12n}(\bar{X}_n - 1/2))$.
    \item Esta é uma distribuição log-normal. Identifique seus parâmetros.
\end{enumerate}

\subsection*{[Questão 39] Teorema da Função Contínua (Distribuição)}

Sejam $U_n \xrightarrow{d} U \sim N(0,1)$ e $V_n \xrightarrow{d} V \sim N(0,1)$ independentes.

\begin{enumerate}[(a)]
    \item Mostre que $(U_n, V_n) \xrightarrow{d} (U, V)$ (convergência conjunta).
    \item Use o teorema da função contínua com $g(u,v) = u^2 + v^2$ para mostrar que $U_n^2 + V_n^2 \xrightarrow{d} \chi^2_2$.
    \item Generalize para $p$ variáveis independentes.
\end{enumerate}

\subsection*{[Questão 40] Teorema da Função Contínua (Distribuição)}

Sejam $X_1, X_2, \ldots, X_n$ v.a.'s i.i.d. com $X_i \sim \text{Poisson}(\lambda)$ onde $\lambda = 10$.

\begin{enumerate}[(a)]
    \item Pelo TCL, $\frac{\sqrt{n}(\bar{X}_n - 10)}{\sqrt{10}} \xrightarrow{d} N(0,1)$.
    \item Defina $Y_n = \sqrt{\bar{X}_n}$. Use o método delta combinado com o teorema da função contínua para analisar a distribuição de $\sqrt{n}(Y_n - \sqrt{10})$.
    \item Compare com a transformação estabilizadora de variância para Poisson.
\end{enumerate}

\section{Estimadores Consistentes}

\subsection*{[Questão 41] Estimadores Consistentes}

Sejam $X_1, X_2, \ldots, X_n$ v.a.'s i.i.d. com $X_i \sim U(0, \theta)$ onde $\theta > 0$.

\begin{enumerate}[(a)]
    \item Seja $T_n = X_{(n)}$ o máximo amostral. Encontre $E[T_n]$ e $\text{Var}(T_n)$.
    \item Mostre que $EQM[T_n] = E[(T_n - \theta)^2] \to 0$ quando $n \to \infty$.
    \item Conclua que $T_n$ é consistente para $\theta$.
    \item Compare com $\hat{\theta}_1 = 2\bar{X}_n$. Qual é mais eficiente assintoticamente?
\end{enumerate}

\subsection*{[Questão 42] Estimadores Consistentes}

Sejam $X_1, X_2, \ldots, X_n$ v.a.'s i.i.d. com $X_i \sim \text{Exp}(\lambda)$.

\begin{enumerate}[(a)]
    \item Considere três estimadores para $\lambda$: 
    \begin{itemize}
        \item $T_n^{(1)} = 1/\bar{X}_n$ (método dos momentos)
        \item $T_n^{(2)} = n/\sum_{i=1}^n X_i$ (EMV)
        \item $T_n^{(3)} = 1/X_{(1)}$ (inverso do mínimo)
    \end{itemize}
    \item Mostre que $T_n^{(1)}$ é consistente usando LFGN e teorema da função contínua.
    \item Mostre que $T_n^{(2)}$ é consistente.
    \item $T_n^{(3)}$ é consistente? Justifique.
\end{enumerate}

\subsection*{[Questão 43] Estimadores Consistentes}

Sejam $X_1, X_2, \ldots, X_n$ v.a.'s i.i.d. com $X_i \sim N(\mu, \sigma^2)$.

\begin{enumerate}[(a)]
    \item Mostre que $S_n^2 = \frac{1}{n-1}\sum_{i=1}^n(X_i - \bar{X}_n)^2$ é consistente para $\sigma^2$.
    \item Mostre que $\tilde{S}_n^2 = \frac{1}{n}\sum_{i=1}^n(X_i - \bar{X}_n)^2$ também é consistente para $\sigma^2$.
    \item Calcule o viés de cada estimador para $n$ finito. Qual você prefere e por quê?
\end{enumerate}

\subsection*{[Questão 44] Estimadores Consistentes}

Sejam $X_1, X_2, \ldots, X_n$ v.a.'s i.i.d. com $X_i \sim \text{Bernoulli}(p)$.

\begin{enumerate}[(a)]
    \item Dado $\varepsilon = 0.05$ e $\delta = 0.01$, encontre o tamanho amostral mínimo $n_0$ tal que
    \[
    P(|\bar{X}_n - p| < \varepsilon) \geq 1 - \delta
    \]
    usando a desigualdade de Chebyshev. Assuma $p = 0.5$ (pior caso).
    \item Refaça usando a aproximação normal (TCL).
    \item Compare os dois valores de $n_0$.
\end{enumerate}

\subsection*{[Questão 45] Estimadores Consistentes}

Sejam $X_1, X_2, \ldots, X_n$ v.a.'s i.i.d. com distribuição desconhecida mas com $E[X_i] = \mu < \infty$ e $\text{Var}(X_i) = \sigma^2 < \infty$.

\begin{enumerate}[(a)]
    \item Defina o estimador "trimmed mean": $\bar{X}_n^{(k)} = \frac{1}{n-2k}\sum_{i=k+1}^{n-k} X_{(i)}$ onde $X_{(1)} \leq \cdots \leq X_{(n)}$ são as estatísticas de ordem, e $k$ é fixo.
    \item Argumente que $\bar{X}_n^{(k)} \xrightarrow{P} \mu$ quando $n \to \infty$ com $k$ fixo.
    \item Discuta a robustez deste estimador comparado a $\bar{X}_n$ na presença de outliers.
\end{enumerate}

\section{Propriedades Assintóticas dos EMVs}

\subsection*{[Questão 46] Propriedades Assintóticas dos EMVs}

Sejam $X_1, X_2, \ldots, X_n$ v.a.'s i.i.d. com $X_i \sim N(\mu, \sigma^2)$ onde $\sigma^2$ é conhecido.

\begin{enumerate}[(a)]
    \item Mostre que o EMV de $\mu$ é $\hat{\mu}_n = \bar{X}_n$.
    \item Calcule a informação de Fisher $I_X(\mu) = E\left[\left(\frac{\partial \log f(X;\mu)}{\partial \mu}\right)^2\right]$.
    \item Use o TCL para EMVs para escrever a distribuição assintótica de $\sqrt{n}(\hat{\mu}_n - \mu)$.
    \item Construa um IC assintótico de 95\% para $\mu$.
\end{enumerate}

\subsection*{[Questão 47] Propriedades Assintóticas dos EMVs}

Sejam $X_1, X_2, \ldots, X_n$ v.a.'s i.i.d. com $X_i \sim \text{Poisson}(\lambda)$.

\begin{enumerate}[(a)]
    \item Mostre que o EMV de $\lambda$ é $\hat{\lambda}_n = \bar{X}_n$.
    \item Calcule a informação de Fisher $I_X(\lambda) = \frac{1}{\lambda}$.
    \item Verifique que $\sqrt{n}(\hat{\lambda}_n - \lambda) \xrightarrow{d} N(0, \lambda)$, confirmando $I_X^{-1}(\lambda) = \lambda$.
    \item Compare com o resultado do TCL clássico.
\end{enumerate}

\subsection*{[Questão 48] Propriedades Assintóticas dos EMVs}

Sejam $X_1, X_2, \ldots, X_n$ v.a.'s i.i.d. com $X_i \sim \text{Exp}(\lambda)$ (taxa $\lambda$).

\begin{enumerate}[(a)]
    \item Mostre que o EMV de $\lambda$ é $\hat{\lambda}_n = \frac{n}{\sum_{i=1}^n X_i} = \frac{1}{\bar{X}_n}$.
    \item Calcule a informação de Fisher $I_X(\lambda) = \frac{1}{\lambda^2}$.
    \item Use o TCL para EMVs para encontrar a distribuição assintótica de $\sqrt{n}(\hat{\lambda}_n - \lambda)$.
    \item Compare este resultado com o obtido usando o método delta aplicado a $1/\bar{X}_n$.
\end{enumerate}

\subsection*{[Questão 49] Propriedades Assintóticas dos EMVs e Eficiência}

Sejam $X_1, X_2, \ldots, X_n$ v.a.'s i.i.d. com $X_i \sim U(0, \theta)$.

\begin{enumerate}[(a)]
    \item O EMV de $\theta$ é $\hat{\theta}_n^{MLE} = X_{(n)}$. O estimador de momentos é $\hat{\theta}_n^{MM} = 2\bar{X}_n$.
    \item Mostre que ambos são consistentes.
    \item Calcule as variâncias assintóticas de $\sqrt{n}(\hat{\theta}_n^{MLE} - \theta)$ e $\sqrt{n}(\hat{\theta}_n^{MM} - \theta)$.
    \item Qual é mais eficiente? Calcule a eficiência relativa assintótica.
    \item Nota: Este é um caso onde as condições de regularidade falham! O EMV não tem distribuição normal assintótica.
\end{enumerate}

\subsection*{[Questão 50] Propriedades Assintóticas dos EMVs}

Sejam $X_1, X_2, \ldots, X_n$ v.a.'s i.i.d. com $X_i \sim \text{Bernoulli}(p)$.

\begin{enumerate}[(a)]
    \item Mostre que o EMV de $p$ é $\hat{p}_n = \bar{X}_n$.
    \item Calcule a informação de Fisher $I_X(p) = \frac{1}{p(1-p)}$.
    \item Use o TCL para EMVs para mostrar que $\sqrt{n}(\hat{p}_n - p) \xrightarrow{d} N(0, p(1-p))$.
    \item Queremos estimar $\psi = \log\left(\frac{p}{1-p}\right)$ (log-odds). Use o método delta para encontrar a distribuição assintótica de $\sqrt{n}(\log(\frac{\hat{p}_n}{1-\hat{p}_n}) - \psi)$.
    \item Verifique que a variância assintótica é $\frac{1}{p(1-p)}$, que é exatamente $I_X^{-1}(p)$ (invariância do EMV sob reparametrização).
\end{enumerate}

\section{Gabarito e Dicas - Parte 2}

\subsection{Dicas para as Novas Questões}

\textbf{TCL para Variância Amostral (Questões 31-35):}
\begin{itemize}
    \item Lembre que $\sqrt{n}(S_n^2 - \sigma^2) \xrightarrow{d} N(0, \mu_4 - \sigma^4)$
    \item Para Normal: $\mu_4 = 3\sigma^4$, logo variância assintótica = $2\sigma^4$
    \item Para Exponencial($\lambda$): $\mu_4 = 9/\lambda^4$, $\sigma^2 = 1/\lambda^2$
    \item Para Uniforme(0,1): $\mu_4 = 1/80$, $\sigma^2 = 1/12$
\end{itemize}

\textbf{Teorema da Função Contínua - Distribuição (Questões 36-40):}
\begin{itemize}
    \item Use quando tiver $U_n \xrightarrow{d} U$ e quiser $g(U_n) \xrightarrow{d} g(U)$
    \item Exemplo clássico: $Z \sim N(0,1) \Rightarrow Z^2 \sim \chi^2_1$
    \item Combine com método delta quando necessário
\end{itemize}

\textbf{Consistência (Questões 41-45):}
\begin{itemize}
    \item Mostre $T_n \xrightarrow{P} \theta$ via: (1) LFGN, (2) EQM $\to$ 0, ou (3) convergência da f.d.a.
    \item Para $X_{(n)}$ em $U(0,\theta)$: $E[X_{(n)}] = \frac{n\theta}{n+1}$, $\text{Var}(X_{(n)}) = \frac{n\theta^2}{(n+1)^2(n+2)}$
    \item Tamanho amostral via Chebyshev: $n \geq \frac{\sigma^2}{\varepsilon^2 \delta}$
\end{itemize}

\textbf{EMVs (Questões 46-50):}
\begin{itemize}
    \item Informação de Fisher: $I_X(\theta) = E\left[\left(\frac{\partial \log f(X;\theta)}{\partial \theta}\right)^2\right] = -E\left[\frac{\partial^2 \log f(X;\theta)}{\partial \theta^2}\right]$
    \item TCL para EMVs: $\sqrt{n}(\hat{\theta}_n - \theta) \xrightarrow{d} N(0, I_X^{-1}(\theta))$
    \item Para Normal($\mu$, $\sigma^2$): $I_X(\mu) = 1/\sigma^2$
    \item Para Poisson($\lambda$): $I_X(\lambda) = 1/\lambda$
    \item Para Exponencial($\lambda$): $I_X(\lambda) = 1/\lambda^2$
    \item Para Bernoulli($p$): $I_X(p) = 1/(p(1-p))$
\end{itemize}

\subsection{Respostas Selecionadas - Parte 2}

\textbf{Questão 31(a):} Para $N(0,1)$: $\mu_4 = E[X^4] = 3$ (use momentos da normal).

\textbf{Questão 36(c):} Este resultado é fundamental para qui-quadrado com 1 grau de liberdade.

\textbf{Questão 41(a):} $E[X_{(n)}] = \frac{n\theta}{n+1}$, $\text{Var}(X_{(n)}) = \frac{n\theta^2}{(n+1)^2(n+2)}$, ambos $\to 0$ quando normalizados.

\textbf{Questão 44(a):} Usando Chebyshev com $p=0.5$: $n_0 \geq \frac{0.25}{(0.05)^2(0.01)} = 10000$.

\textbf{Questão 44(b):} Usando TCL: $n_0 = \left\lceil \left(\frac{z_{0.005}\sqrt{0.25}}{0.05}\right)^2 \right\rceil = \left\lceil (51.5)^2 \right\rceil = 2653$.

\textbf{Questão 49(e):} Caso especial onde condições de regularidade falham. O EMV converge mais rápido ($n$ ao invés de $\sqrt{n}$).

\textbf{Questão 50(e):} Demonstração da invariância da informação de Fisher sob transformações.

\end{document}

