\begin{equation}
\sqrt{n} \left[ T_n^{(2)} - g(\theta) \right] \xrightarrow[n \to \infty]{d} N\left(0, \sigma_2^2(\theta)\right),
\end{equation}

então a eficiência relativa assintótica de $T^{(2)}$ com respeito a $T^{(1)}$ é definida como

\begin{equation}
\frac{\sigma_1^2(\theta)}{\sigma_2^2(\theta)}
\end{equation}

Os EMVs são assintoticamente eficientes pelo teorema abaixo.

\textbf{Teorema (3.8.1) [TCL para os EMVs]}

Sejam $X_1, \ldots, X_n$ uma amostra de $X$ com fdp (ou fmp) $f(x; \theta)$ para $x \in \mathbb{X} \subset \mathbb{R}$ e $\theta \in \Theta \subset \mathbb{R}$ tal que $\Theta$ é um intervalo aberto.

Assuma que:

\textbf{(A1)} $\theta \mapsto f(x; \theta)$ é três vezes diferenciável sobre $\Theta$, $\forall x \in \mathbb{X}$.

\textbf{(A2)} 
\begin{equation}
\int_{\mathbb{X}} \frac{\partial}{\partial \theta} f(x; \theta) \, dx = 0 \quad \left( = \frac{\partial}{\partial \theta} \int_{\mathbb{X}} f(x; \theta) \, dx \right)
\end{equation}
e
\begin{equation}
\int_{\mathbb{X}} \frac{\partial^2}{\partial \theta^2} f(x; \theta) \, dx = 0 \quad \left( = \frac{\partial^2}{\partial \theta^2} \int_{\mathbb{X}} f(x; \theta) \, dx \right)
\end{equation}